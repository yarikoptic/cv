\documentclass[10pt,overlapped,line]{res}
%\documentstyle[margin, line]{res}


%\usepackage{url}
%for funny lists
\usepackage{pifont}		
\usepackage{multicol}
\usepackage{pstricks,pst-node,pst-text,pst-3d}
\usepackage{amsmath}
\usepackage{pslatex}

\usepackage[dvips]{graphicx}
\usepackage[usenames,dvipsnames]{color}
\usepackage[]{hyperref}



\def\Cplusplus{{\rm C\raise.5ex\hbox{\small ++}}}
\newcommand{\mplace}[1]{\textbf{#1}}
\newcommand{\wdescription}[1]{({\small \textit{#1}})}
\newcommand{\mtitle}[1]{``#1''}
\newcommand{\mauthors}[1]{ \textit{#1.}}
\newcommand{\mwhere}[1]{#1.}
\oddsidemargin -.5in
\evensidemargin -.5in
\textwidth=7.0in

\begin{document}
\name{Yaroslav Halchenko}
\address{ 
  \textbf{Preferred Mailing Address:}\newline 
  Computer Science Dept., FEC \\ 
  UNM, Albuquerque NM, 87131 \\  
  USA \newline
}  
\address{
   \textbf{Current Address}:\newline 
   1335 Ortiz SE.  Apt.13 \\
   Albuquerque NM, 87108 \\
   USA
}
\address{  
  \href{URL}{email:yoh@cs.unm.edu}\\ 
  \href{URL}{http://www.onerussian.com}\\
  \texttt{(505) 266-5958 (home)}
} 
 
\begin{resume}

\section{Objective} 
   To combine theoretical knowledge (math, physics, numeric methods, etc)
   and computer programming skills to produce ideas and results useful
   for the public.
   
\section{Education}
\begin{format}
  \title{l}\dates{r}\\
  \employer{l}\location{r}\\
  \body\\
\end{format}

\employer{Computer Science Dept, UNM (University of New Mexico)}
\title{Ph.D. student}
\location{Albuquerque NM, USA}
\dates{January 2000 -- present}
\begin{position}
  Graduate Ph.D. program student with overall GPA 4.0. Planning to
take comprehensive examination in Fall semester of 2002.
\end{position}

\employer{VSTU (Vinnytsia State Technical University)}
\title{Graduate student}
\location{Vinnytsia, Ukraine}
\dates{1998 -- 1999}
\begin{position}
 Master`s degree in Laser and Optoelectronic Engineering
\end{position}

\employer{VSTU}
\title{Student}
\location{Vinnytsia, Ukraine}
\dates{1994 -- 1998}
\begin{position}
 Bachelor's degree honor diploma in Laser and Optoelectronic Engineering.
\end{position}

\employer{Physical and Mathematical Gymnasia No.17}
\title{High School Student}
\location{Vinnytsia, Ukraine}
\dates{1992 -- 1994}
\begin{position}
 Graduated with honors.
\end{position}

\employer{Physical and Mathematical Correspondence School}
\title{High School Student}
\location{Moscow Physical Engineering Institute}
\dates{1991 -- 1994}
\begin{position}
\end{position}


\employer{Aerospace Correspondence School ``Soyuz''} 
\title{High School Student}
\location{Moscow Physical Engineering Institute}
\dates{1991 -- 1992}
\begin{position}
\end{position}
 

\section{Work Experience}
\begin{format}
  \title{l}\dates{r}\\
  \employer{l}\location{r}\\
  \body\\
\end{format}


\employer{Brain and Computation Lab}
\title{Research Assistant}
\location{CS Dept., UNM Albuquerque NM, USA}
\dates{June 2000 -- present}
\begin{position}
 Have been working\footnote{\href{URL}{http://www-bcl.cs.unm.edu}} on deploying ICA\footnote{Independent Component Analysis} techniques in processing of
MEG\footnote{MagnetoEncephaloGraphy} data. 
\end{position}

\employer{CS Dept., UNM} 
\title{Teaching Assistant}
\location{Albuquerque NM, USA}
\dates{January 2000 -- May 2000}
\begin{position}
  TA for ``Intermediate Programming'' (CS251) class with David
  Ackley\footnote{ackley@cs.unm.edu}
\end{position}


\employer{VSTU}
\title{Research Assistant}
\location{Vinnitsya, Ukraine}
\dates{ 1993 -- 1997 }
\begin{position}
Part-time designer and software developer of the system for
diagnostics of mobile segments in vertebral column, which later was
utilized in national hospitals of Ukraine.
\end{position}

\employer{Liana Company}
\title{Software Developer}
\location{Vinnitsya, Ukraine}
\dates{ April 1996 -- October 1997 }
\begin{position}
Part-time software developer of automated system in
Planned-Economic Department of Vinnytsia Chemical Plant (Himprom).
\end{position}

\section{Scholarship Awards}
\begin{format}
  \employer{l} 
  \dates{r}\\ 
  \body\\
\end{format}

\employer{} 
\dates{ 1998-1999 }
\begin{position}
 The International Scientific Fund Representatives in Ukrainian Studentship Award. 
\end{position}

\employer{} 
\dates{ 1995-1996 }
\begin{position}
 The International Soros Science Educational Program (ISSEP) Studentship Award
\end{position}


\section{Main Awards and Honors}
\begin{format}
  \employer{l} \dates{r}\\ 
  \body\\
\end{format}

\employer{Ukraine}
\dates{ 1996 }
\begin{position}
  The Academy of Sciences of Ukraine \mplace{awarded the work}
  ``Information-Measuring System With Optical Transformation
  Biomedical Information''.
\end{position}

\employer{Bucharest Romania}
\dates{ October 1995 }
\begin{position}
  \mplace{6th place} in ACM South-Eastern European Regional
  Programming Contest. 1st place at VSTU.
\end{position}

\employer{Kharkiv, Ukraine}
\dates{ April 1995 }
\begin{position}
  \mplace{4th place} in Physics Olympiad among Colleges and
  Universities of Ukraine. \mplace{1st place} at VSTU.
\end{position}

\employer{ Vinnitsya, Ukraine }
\dates{ 1994 }
\begin{position}
\mplace{1st place} in the Regional Contest for Programming.
\end{position}

\employer{ Ukraine }
\dates{ 1994 }
\begin{position}
  \mplace{1st place} in Competition among teenagers for the best
  computer program. Became a \mplace{Member of the Ukraine Small Academy
    of Sciences}.
\end{position}

\employer{Vinnitsya, Ukraine}
\dates{ 1993 }
\begin{position}
  \mplace{1st place} for the best solution of physical and mathematical
  problems in the competition organized by Moscow Physical Engineering Institute.
\end{position}

\employer{Vinnitsya, Ukraine}
\dates{ 1993 }
\begin{position}
  \mplace{3rd place} in the Regional Physics Olympiad.

\end{position}

\section{Computer Skills}
 \begin{description}
   \item[Programming]:\\
     \begin{itemize}

     \item Software developer under Linux with over 2 years of
       experience (C/\Cplusplus(gcc), bash, perl, CVS). 
       \wdescription{Currently involved in development process of
       DreamMon project holded by BCL\footnote{ Brain and Computation
       Lab \href{URL}{http://www-bcl.cs.unm.edu}}}

     \item Knowledge of logic and functional languages (prolog, SML, e-lisp).
     \item Basic knowledge of Java programming (Swing, RMI, JDBC).
     \item Software developer with over 6 years of experience on DOS,
       Windows 3.x and Windows 9x platforms.
       \wdescription{Have written number of small tools and couple of
       projects. Refer to ``Work Experience''}
     \item Experienced with Visual Basic for Applications and Inprise
       Delphi.
     \item Database design. Working knowledge of DBE, ODBC, Postgresql
       APIs. 
     \item Strong background in object-oriented programming methods
       and Design Patterns.
     \item Experienced writer of good quality code. Coding practice
       includes thorough code reviews, careful debugging, detecting
       bugs, and other techniques.
     \end{itemize} 

   \item[Administration]:\\ 
     \begin{itemize}
     \item Linux-based (GNU Debian) computer and network (TCP/IP,
       SNMP, DNS, BIND, NAT, NFS, AMD, SSH, Sambda, X Windows) maintaining. 
       \wdescription{Maintaining computer park at BCL}
     \item Apache web site administration.
       \wdescription{Still running on \href{URL}{http://www.onerussian.com}}
     \item DB administration (postgresql).
     \end{itemize}

   \item[Web Design and Others]:\\
     \begin{itemize}
     \item Basic web-designer skills (html, css).
       \wdescription{Example is my home page \href{URL}{http://www.onerussian.com}}
     \item Strong background in graphical design and desktop
       publishing (\LaTeX\ , Adobe Photoshop, Visio, Microsoft Office, ABBYY
       FineReader etc).
     \item Excellent knowledge of mathematics and numerical methods
       for computer aided design. Excellent knowledge of MathCad and
       MathLab.
     \item Work experience with Electronic Design Automation: PCAD and
       Visio.
     \end{itemize}
 \end{description}

\section{Publications}

\begin{itemize}
 
 \item 
    \mauthors{Timchenko L. I., Kutaev Y. F., Gertsiy A. A., Halchenko Y. O.}
    \mtitle{Method for image coordinate definition on extended laser paths} 
    \mwhere{Optoelectronic and Hybrid Optical/Digital Systems for Image and Signal Processing, Published June 2000,  Volume 4148-19}

 \item 
    \mauthors{L.I.Timchenko, Y.F.Kutaev, A.A.Gertsiy, Y.O.Halchenko, M.A.Grudin} 
    \mtitle{Approach to parallel-hierarchical network learning for real-time image sequence recognition} 
    \mwhere{The International Symposium on Intelligent Systems and Advanced Manufacturing, 19-22 September 1999, Massachusetts USA. Volume 3836-09}

  \item 
    \mauthors{L. Timchenco, Y.Kutaev, A.Gertsiy, L. Zagoruiko, Y. Halchenko} 
    \mtitle{Pre-processing of extended laser path images} 
    \mwhere{Industrial Lasers and Inspection, EOS/SPIE International Symposium. Munich, 14-18 June 1999. Volume 3827-26}

  \item 
    \mauthors{Leonid Tymchenko, Janina Scorukova, Serhij Markov, Yaroslav Halchenko} 
    \mtitle{Image Segmentation on the basis of spatial connected features} 
    \mwhere{Visnyk VSTU, No. 4, pp. 39-43, Ukraine, in Ukrainian, 1998}

  \item 
    \mauthors{T.B.Martinyuk, A.V.Kogemiako, Y.O.Halchenko } 
    \mtitle{The model of associative processor for numerical data sorting} 
    \mwhere{ Visnyk VSTU. No. 2, pp. 19-23, Ukraine, in Ukrainian, 1997}

  \item 
    \mauthors{Leonid Tymchenko, Janina Scorukova, Jurij Kutaev, Serhij Markov, Tatiana Martynuk, Yaroslav Halchenko} 
    \mtitle{Method Spatial Connected Segmentation of Images}
    \mwhere{The Third All-Ukrainian International Conference Ukrobraz, Kijiv, Ukraine, November 26-30, 1996}
\end{itemize}

Other presentations of my projects have been presented to research visitors and at local conferences

\section{International Skills}
 \begin{description}
 \item[Languages]: Russian, Ukrainian and English. 
 \item[Citizenship]: Ukraine
 \end{description}
\section{References}
 \begin{multicols}{2}
{\small
%   Dr. Barak A. Pearlmutter, Assistant Professor\\
%   Department of Computer Science \\
%   Farris Engineering Center \\
%   University of New Mexico \\
%   Albuquerque, NM 87131 \\
%   Tel: (505) 277-9738 office \\
%   Tel: (505) 277-1592 lab \\
%   \href{URL}{bap@cs.unm.edu} \\
   Prof Barak A. Pearlmutter, PhD \\
   Department of Computer Science \\
   FEC 157 \\
   University of New Mexico \\
   Albuquerque, NM  87131 \\
   Tel: (505) 277-3112  \\
   \href{URL}{bap@cs.unm.edu} \\

   Dr. Akaysha Tang, Assistant Professor\\
   Department of Psychology \\
   Logan Hall \\
   University of New Mexico \\
   Albuquerque, NM 87131 \\
   Tel: (505) 277-4025 office \\
   Tel: (505) 277-5397/4931 lab \\
   \href{URL}{akaysha@unm.edu} \\
}
\end{multicols}

Additional references available upon request.



\end{resume}
\end{document}
