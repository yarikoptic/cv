% rubber: module pdftex
% halchenko_vita.tex
% Academic vita for Yaroslav O. Halchenko
% Based on the "ModernCV" template by Xavier Danaux
% which was adopted by Dylan D. Wagner and served me an example
% License: http://creativecommons.org/licenses/by-nc-sa/3.0/
%=============================================================
%---Document Class
\documentclass[11pt,letterpaper,resume,roman]{moderncv}
% Font sizes: 10, 11, or 12; 
% paper sizes: a4paper, letterpaper, a5paper, legalpaper, executivepaper or landscape; 
% font families: sans or roman


%---Configure moderncv style
\moderncvstyle{classic}
% CV theme - options include: 'casual' (default), 'classic', 'oldstyle' and 'banking'

\moderncvcolor{blue} 
% CV color - options include: 'blue' (default), 'orange', 'green',
%                             'red', 'purple', 'grey' and 'black'
\setlength{\hintscolumnwidth}{2.5cm} % Sets the width of the dates column
%\setlength{\makecvtitlenamewidth}{10cm} 
% For the 'classic' style, adjusts the width of the space allocated to name

\usepackage{moderntimeline}
\tlmaxdates{1994}{2015}
%\tlwidth{0.8ex}
% Set the labels text size
%\tltext{\tiny}
%\tltextstart[south east]{\scriptsize}
%\tltextend[south west]{\scriptsize}

% both seems to be of no effect :-/

%\usepackage{hyperref}
% \newcommand\Colorhref[3][cyan]{\href{#2}{\small\color{#1}#3}}
\AtBeginDocument{\hypersetup{pdfborder = 0 0 1,linkcolor=blue}}

\usepackage{xstring}
\def\nohttp#1{%
    \StrSubstitute{#1}{http://}{}{}}
%\newcommand{\pname}[1]{#1}
%\newcommand{\purl}[2]{\pname{#1}\footnote{\url{#2}}}
% more definetly to use hyperref -- so just a convenience to avoid
% swapping #1 and #2 at every occurance;-)
%\newcommand{\puurl}[2]{\href{#2}{\mbox{#1}}}
% And this one is the first occurrence ones which we make both a hyperlink and an explicit url
%\newcommand{\phref}[2]{\href{#2}{#1} (\href{#2}{\nohttp{#2}})}
\newcommand{\phref}[2]{#2 (\href{#1}{#1})}
\newcommand{\phttphref}[2]{#2 (\href{http://#1}{#1})}
\newcommand{\phttp}[1]{\href{http://#1}{#1}}


\usepackage[]{natbib}           % To make workaround work
% workaround for bibentry + hyperref
\makeatletter
\let\saved@bibitem\@bibitem
\makeatother
\usepackage{bibentry}           % To automate inclusion of references
                                % separately in 2 sections

% yoh: dislikes automatic commas
\renewcommand*{\cventry}[7][.25em]{%
  \cvitem[#1]{#2}{%
    {\bfseries#3}%
    \ifthenelse{\equal{#4}{}}{}{\ifthenelse{\equal{#3}{}}{}{,} {\slshape#4}}%
    \ifthenelse{\equal{#5}{}}{}{, #5}%
    \ifthenelse{\equal{#6}{}}{}{, #6}%
    \ifthenelse{\equal{#4#5#6}{}}{}{.\strut}%
    \ifx&#7&%
      \else{\newline{}\begin{minipage}[t]{\linewidth}\small#7\end{minipage}}\fi}}

\newcommand{\hurl}[1]{\href{http://#1}{#1}}
\def\Cplusplus{C++} % {\rm C\raise.5ex\hbox{\small ++}}}

\usepackage[papersize={8.5in,11in},
left=0.5in, right=0.5in, top=0.5in, bottom=0.7in]{geometry}

%---Load packages
%\usepackage[]{a4wide}   %Wider margins
\usepackage[]{times}    %Nicer fonts
% Provides special list environments and macros to create new ones
\usepackage[shortlabels]{enumitem}
%\usepackage{doi}

%---Information for makecvtitle
\firstname{Yaroslav O.} % Your first name
\familyname{Halchenko} % Your last name  , Ph.D.
% All information in this block is optional, comment out any lines you don't need
%\title{Department of Psychological and Brain Sciences, Dartmouth College}
\address{6207 Moore Hall}{Dartmouth College}
\mobile{Hanover, NH 03755}
\phone{(603) 646-9834}
\fax{(603) 398-1138}
\email{career@onerussian.com}
\homepage{haxbylab.dartmouth.edu/ppl/yarik.html}%{haxbylab.dartmouth.edu/ppl/yarik.html} % The first argument is the url for the clickable link, the second argument is the url displayed in the template - this allows special characters to be displayed such as the tilde in this example
\extrainfo{
  \hurl{www.linkedin.com/in/yarik}
  \hurl{github.com/yarikoptic}
%  \hurl{ohloh.net/accounts/yarikoptic}
 }
% \hurl{plus.google.com/+YaroslavHalchenko}}

%can use newline in cventry items to space things out. 
%\newline{}\newline{}


% %Stuff for bib
 \makeatletter
%  \newlength{\bibhang}
  \setlength{\bibhang}{0em}
%  \newlength{\bibsep}
%   {\@listi \global\bibsep\itemsep \global\advance\bibsep by\parsep}
%  \newlist{bibsection}{itemize}{3}
%  \setlist[bibsection]{label=,leftmargin=\bibhang,%
%          itemindent=-\bibhang,
%          itemsep=\bibsep,parsep=\z@,partopsep=0pt,
%          topsep=0pt}
  \newlist{bibenum}{enumerate}{3}
  \setlist[bibenum]{label=[\arabic*],resume,leftmargin={2.8cm},%
          itemindent=-\bibhang,
          itemsep=\bibsep,parsep=\z@,partopsep=0pt,
          topsep=0pt}
  \let\oldendbibenum\endbibenum
  \def\endbibenum{\oldendbibenum\vspace{-.6\baselineskip}}
  \let\oldendbibsection\endbibsection
  \def\endbibsection{\oldendbibsection\vspace{-.6\baselineskip}}
  \makeatother

\newcommand{\ie}[0]{\emph{i.e.},\ }
\newcommand{\eg}[0]{\emph{e.g.},\ }
\newcommand{\etc}[0]{\emph{etc.}}

% 
%----------------------------------------------------------------------------------------
\begin{document}
\makecvtitle % Print the CV title

% ideas: Professional network
%\cvline{Linkedin.com}{\small \href{http://www.linkedin.com/in/adamatan}             {Adam Matan}   - Professional profile and links.    }
%\cvline{Stackoverflow.com}{\small \href{http://stackoverflow.com/users/51197/adam-matan} {Adam Matan}   - My software questions and answers. }
%\cvline{twitter.com}{\small \href{http://twitter.com/justnoticed}

%---Education
\vspace{-2em}
\section{Education and Training}
\tllabelcventry{2009}{2012}{\hspace{-1.5em}2012-2013}{Postdoctoral Fellow}{}{}{}{Department of Psychological \& Brain Sciences, Dartmouth College\newline Adviser: \href{mailto:James.V.Haxby@dartmouth.edu}{Dr. James V. Haxby}}
\tllabelcventry{2004}{2009}{\hspace{-1em}2004-2009}{Ph.D. in Computer Science}{}{}{}{Computer Science Department, NJIT (NJ Institute of Technology)\newline Adviser: \href{mailto:jose@rubic.rutgers.edu}{Dr. Stephen J. Hanson}, Rutgers-Newark}
\tllabelcventry{2000}{2003}{\hspace{-1.5em}2000-2003}{M.S. in Computer Science}{}{}{}{Computer Science Department, UNM (University of New Mexico)\newline Adviser: \href{mailto:barak@cs.nuim.ie}{Dr. Barak Pearlmutter}}
\tllabelcventry{1994}{1999}{1999}{M.S. in Laser and Optoelectronic Engineering}{}{}{}{Computer Systems Department, VSTU (Vinnytsia State Technical University), Ukraine}
\tldatecventry{1994}{Graduated with honors}{}{}{}{Physics and Mathematical Gymnasia No.17, Ukraine}

% looks like a neat list
%\cvlistdoubleitem{Mathematical Logic I}{Mathematical Logic II}
%\cvlistdoubleitem{Modal Logic}{Intensional and Higher-Order Logic}
%\cvlistdoubleitem{Philosophy of Mathematics}{Early Analytic Philosophy}

%---Employment
% TODO: add to each role
%  "Skillset" summarizing the technologies/approaches
%  "References"
% Customize item lists to reduce spacing a bit
\section{Employment}

%\cventry{2013--present}
\tlcventry{2013}{0}{Research Scientist}{}{}{Department of Psychological \& Brain Sciences, Dartmouth College}{
\begin{itemize}[leftmargin=1em,parsep=0em]
\item Visual perception: effects of familiarity on face identification
  [\ref{GGH+13}, \ref{GGH+13b}]
\item Participating in lab's methodological developments for
  neuroimaging data analysis: hyperalignment [\ref{HGC+11}], RSA [\ref{CGG+12}], clustering, \etc
\item Software framework for statistical analysis of neural data:
  \phttphref{pymvpa.org}{PyMVPA} [\ref{HHS+09a}, \ref{HHS+09b}]
\item Turnkey software platform for neuroscience: \phttphref{neuro.debian.net}{NeuroDebian} [\ref{HH12}]
\end{itemize}
}

\tllabelcventry{2005}{2009}{\hspace{-1.5em}2005-2009}{Computing Cluster System Administrator}{}{}{Rutgers-Newark, NJ}{
%  \begin{itemize}
%  \item 
Deployment and maintenance of 27 node high availability cluster running GNU/Linux Debian OS
%  \end{itemize}
}

\tllabelcventry{2003}{2009}{\hspace{-1.2em}2003-2009}{Research Assistant}{}{}%
{\href{http://www.rumba.rutgers.edu}{Mind/Brain RUMBA Laboratory}, Rutgers-Newark, NJ}{
   \begin{itemize}[leftmargin=1em,parsep=0em]
   \item Predictive decoding and fusion of the neural data from and
     across different imaging modalities (e.g. EEG, fMRI) to gain
     better understanding of perception (e.g. auditory) and cognitive
     (e.g. category specific processing) neuroscientific problems [\ref{PHH09}-\ref{HRH+07}]
   \item Graphical modeling of functional brain organization [\ref{RHH+10}]
   \end{itemize}
}

\tllabelcventry{2000}{2002}{\hspace{-1.5em}2000-2002}{Research Assistant}{}{}
{\href{http://www-bcl.cs.unm.edu}{Brain and Computation Laboratory}, UNM Albuquerque, NM}{
  %\begin{itemize}
  %\item 
   Implementation and deployment of ICA (Independent Component
   Analysis) techniques for processing of MEG (Magnetoencephalography)
   data as a part of the DreamMon project
  %\end{itemize}
}

\tllabelcventry{1996}{1997}{\hspace{-1.5em}1996-1997}{Software Developer}{}{}%
{%\phttphref{www-bcl.cs.unm.edu}{Brain and Computation Laboratory},
 \href{http://www.webliana.com.ua}{Liana Company}, Vinnytsia, Ukraine}{
%  \begin{itemize}
%  \item
 Automated system for Planned-Economic Department of Vinnytsia Chemical Plant (Himprom)
%  \end{itemize}
}


\tllabelcventry{1993}{1997}{1993--1997}{Research Assistant}{}{}{}%
{VSTU, Vinnytsia, Ukraine\\
  %\begin{itemize}
  %\item Software developer of the 
   System for diagnostics of vertebral column. System later was utilized in national hospitals of
    Ukraine
%  \end{itemize}
}

\section{Technical Skills}

\cvitem{Programming}{\vspace{-2em}
\begin{itemize}[leftmargin=1em,parsep=0em]
\item More than 13 years of experience with software
  development under GNU/Linux OS:  Python, shell
  scripting, version control systems (CVS, subversion, git,
  git-annex), debugging (gdb, pdb, bashdb, ddd), troubleshooting
  (valgrind, strace), profiling, \emph{etc.}
\item Years of use and contributions to a wide-range of Python
  libraries for generic (\eg\ NumPy, SciPy, sklearn, pandas,
  statsmodels) and neuroimaging-oriented (\eg nibabel, nipy, nipype)
  scientific Python libraries
\item Experience with generic build frameworks (make, cmake),
  continuous integration platforms (\emph{e.g.} buildbot,
  \href{http://travis-ci.org}{Travis})
\item Varying programming experience in other functional (ELisp,
  Standard ML) and imperative (C/\Cplusplus(g++), Java, JavaScript,
  Perl, PHP) languages, and computational environments (Matlab/Octave)
\item Past working experience in software development on MS DOS and
  Windows Platforms (Turbo Pascal, VBA, Inprise Delphi) and Database
  design (DBE, ODBC, Postresql, MySQL)
\item Strong background in object-oriented programming methods
  and Design Patterns
\item Experienced writer of high quality well documented code. Coding
  practice includes peer programming, code reviews, careful
  troubleshooting and debugging of own code and code of others, bug
  triaging, profiling, versioning, unit-, doc- and regression testing,
  release management
\end{itemize}
}
\vspace{-1em}
\cvitem{Systems Administration}{\vspace{-2em}
\begin{itemize}[leftmargin=1em,parsep=0em]
\item Servers and high throughput clusters administration and
  monitoring (DNS, NFS, SSH, NAT, Torque, Ganglia, Maui, SGE,
  HTCondor)
\item Automated provisioning of bare and virtualized deployments
  (Debian FAI, cfengine2, Ansible)
\end{itemize}
}

\section{Grant Proposals Writing}

\cvitem{PI/Co-PI}{Composed and submitted two NSF and three NIH (R01) proposals}
\cvitem{Sub-contract PI}{Participated in two BD2K NIH proposals}
\cvitem{Pre-app.}{Moore foundation, NSF BRAIN EAGER}

\section{Professional Activities}

\subsection*{SERVICE \& OUTREACH}
\tlcventry{2013}{0}{Contributor}{}{\phttphref{nipy.bic.berkeley.edu}{Nibotmi}}{}%
{
  Continuous integration (CI) service initiated by Matthew Brett (UC
  Berkeley) to solidify quality assurance of scientific Python
  projects.  My contribution is in establishing CI for various
  projects (\eg\ sklearn, pandas) with accent on testing on exotic
  hardware platforms such as UltraSPARC}

\tlcventry{2013}{0}{Founder/Leading Developer}{}{\phttphref{yarikoptic.github.io/numpy-vbench}{NumPy Benchmarking}}{}%
{
    \href{http://numpy.org}{NumPy} is the core computational library used by Python community.
    I have established a service continuously benchmarking NumPy
    functionality across different development branches to guarantee
    absent performance regressions
}

\tlcventry{2011}{0}{Initiator}{}{\phttphref{github.com/nipy/nipy-artwork}{NiPy Artwork}}{}%
{%
    Promotional and informative materials for Python-based scientific software
    projects in
    \href{http://www.onerussian.com/tmp/nipy-handout.pdf}{neuroimaging}
    and
    \href{http://www.onerussian.com/tmp/eppy-handout.pdf}{electrophysiology}%
}

\tlcventry{2007}{0}{Founder/Leading Developer}{}{\phttphref{www.pymvpa.org}{PyMVPA}}{}%
{ A Python framework to streamline application of classical and novel
  statistical learning methods for the analysis of neural data.
  PyMVPA has a world-wide user base and empowered numerous studies
  (see \phttp{www.pymvpa.org/whoisusingit.html})
}

\tlcventry{2007}{0}{Founder/Leading Developer}{}{\phttphref{neuro.debian.net}{NeuroDebian}}{}%
{ %\begin{itemize}[leftmargin=1em,parsep=0em]
  NeuroDebian project builds atop of Debian to provide scientific
  community with a turnkey Free and Open-source Software (FOSS)
  platform for neuroscience (and beyond) [\ref{HH12},
  \ref{SFN13}-\ref{INCF11}]
  \begin{itemize}[leftmargin=1em,parsep=-0.2em]
    \item Consulting FOSS projects on aspects of legal assurance
      (copyright/licenses), deployment, and quality assurance
    \item Integrating and maintaining (scientific) free and
      open-source software within the Debian GNU/Linux OS (AFNI,
      nibabel, nipy, PsychoPy, \etc)
    \item Mentoring and sponsoring uploads of contributions
      (OpenSesame, Stimfit, OpenWalnut, \etc) to Debian and
      NeuroDebian repositories
%    \item Provisioning VirtualBox appliance for NeuroDebian
%      deployments on Windows and OSX
    \end{itemize}
    \vspace{-0.5em}
  %\citep{HH12,SFN13,FLOSS4SCIENCE11,INCF11} for more information.
  \begin{description}[leftmargin=1em,parsep=0em]
  \item[Popularity] Complete number of ``downloads'' or installations of
    NeuroDebian-maintained software is impossible to assess because
    majority of packages is also uploaded to official Debian
    distribution and thus made available from any of its more than 130
    derivative distributions (such as Ubuntu).  Main NeuroDebian website
    is accessed by more than 13,000 unique IPs each month, is mirrored
    by 8 contributors world-wide, and receives more than 700 of periodic
    \href{http://neuro.debian.net/popularity.html\#popularity-contest}{voluntary
      ``popularity contest'' submissions}.
    See also \phttphref{neuro.debian.net/testimonials.html}{NeuroDebian users' testimonials}
%  \item[Activities]\hspace{1em}
  \item[Outreach] Since 2010 hosted booth exhibits at annual meetings of
    Society for Neuroscience and Organization for Human Brain Mapping
  \end{description}
}

\tlcventry{2005}{0}{Developer}{}{\phttphref{www.debian.org}{Debian Project}}{}{
  A widely popular community-driven GNU/Linux distribution with over a hundred of
  derivative distributions and millions of users
}

\tlcventry{2005}{0}{Leading Developer}{}{\phttphref{www.fail2ban.org}{Fail2Ban Project}}{}{
  A popular intrusion prevention system possibly having millions of users
}

\tlcventry{2004}{0}{FOSS Contributor}{}{}{}%
{ I have contributed minor fixes and improvements to nearly a hundred
  of FOSS projects.  See \phttp{www.ohloh.net/accounts/yarikoptic} for an overview.
}


\vspace{1em}

%\subsection*{EXHIBITING EXPERIENCE}
%\tlcventry{2010}{0}{}{\href{http://neuro.debian.net}{NeuroDebian}}{}{}{
%Booth exhibits at Society for Neuroscience and Organization for Human Brain Mapping meetings}{}

%\cvitem{AD-HOC REVIEWER}{}
\subsection*{EDITING AND REVIEWING}
\cvitem{Guest Editor}{\href{http://www.frontiersin.org/neuroinformatics/researchtopics/python_in_neuroscience_ii/1591}{Python
    in Neuroscience II} special issue, Frontiers in Neuroscience \& Brain Imaging Methods}

%\subsection*{REVIEW EDITOR}
\cvitem{Review editor}{Frontiers in Brain Imaging Research}

\cvitem{Ad-hoc reviewer for journals}{
Brain Structure and Function,
Frontiers in Neuroinformatics,
Human Brain Mapping,
IEEE Transactions on Signal Processing,
Journal of Cognitive Neuroscience,
Journal of Machine Learning Research,
Nature's Scientific Data,
Neural Computation,
NeuroImage,
Neuroreport,
Pattern Recognition,
SPIE
}

\cvitem{Conference Abstracts}{NIPS, SciPy}

\subsection*{MEMBERSHIPS}

\cvitem{Active}{
  \href{https://www.python.org/psf}{Python Software Foundation}, INCF \href{http://www.incf.org/programs/datasharing/neuroimaging-task-force}{Standards for Data Sharing (Neuroimaging taskforce)}
}
%\subsection*{PAST AND CURRENT SOCIETY MEMBERSHIPS}
\cvitem{Past}{
Association for Psychological Science,
Organization for Human Brain Mapping,
Society for Neuroscience,
Ukraine Small Academy of Sciences
}

%\subsection*{GUEST EDITOR}

\section{Publications}

% TODO: add a reference to Google scholar and state h-index
%\cvitem{\href{http://scholar.google.com/citations?user=EbtfZcwAAAAJ}{Google
%    Scholar}}{h-index: 11 \hspace{2em} i10-index: 12}

\bibliographystyle{abbrvnat}    % also puts URLS out
%\bibliographystyle{apalike}

% Workaround for bibentry + hyperref
 \begingroup
    \makeatletter
    \let\@bibitem\saved@bibitem
    \nobibliography{haxbylab}
  \endgroup

\subsection{SELECTED PEER-REVIEWED ARTICLES (\href{http://scholar.google.com/citations?user=EbtfZcwAAAAJ}{Google
    Scholar h-index}: 11)}

% useful DOI
\renewcommand{\doi}[1]{{doi:~\href{http://dx.doi.org/#1}{#1}}}
\renewcommand{\url}[1]{\href{#1}{#1}}
\newcommand{\rbibentry}[1]{\label{#1}\bibentry{#1}} % so we could
                                % reference them. natbib doesn't know
                                % the order for some reason

\sloppy
\begin{bibenum}[ref=\arabic*,parsep=-0.4em]
%\begin{enumerate}[leftmargin=7em,parsep=0em]
\item \rbibentry{GGH+13}         % Processing of invisible social cues
\item \rbibentry{KFR+13}         % Pattern classification precedes ...
\item \rbibentry{GGH+13b}        % Prioritized Detection of Personally Familiar Faces
\item \rbibentry{CGG+12}         % Representation of biological classes in the human brain
\item \rbibentry{HH12}           % Open is not enough. ...
\item \rbibentry{PBG+12}         % Data sharing in neuroimaging research
\item \rbibentry{GBM+11}         % Nipype: a flexible, lightweight  ...
\item \rbibentry{HH11}           % Neuroscience runs on GNU/Linux
\item \rbibentry{HGC+11}         % A Common, High-Dimensional Model ...
\item \rbibentry{HHH+10}         % Statistical learning analysis in neuroscience: aiming for transparency
\item \rbibentry{RHH+10}         % Six problems for causal inference from fMRI
\item \rbibentry{HHS+09a}        % PyMVPA: A Python toolbox for multivariate pattern analysis of fMRI data
\item \rbibentry{HHS+09b}        % PyMVPA: A Unifying Approach to the Analysis of Neuroscientific Data
\item \rbibentry{PHH09}          % Decoding the Large-Scale Structure of Brain Function ...
\item \rbibentry{HH08}           % Brain reading using full brain support vector machines  ...
\item \rbibentry{HHH+07}         % Bottom-up and top-down brain functional ...
\item \rbibentry{HRH+07}         % Dense mode clustering in brain maps
%\item \rbibentry{TKG+00}         % Method for image coordinate definition on extended laser paths
%\item \rbibentry{TKG+99}         % Approach to parallel-hierarchical network ...
%\item \rbibentry{TSM+98}         % Image Segmentation on the basis of spatial connected features
%\item \rbibentry{MKH97}          % The model of associative processor for numerical data sorting
\end{bibenum}
\vspace{1em}

\subsection{PH.D. THESIS}

\begin{bibenum}[ref=\arabic*,parsep=-0.4em]
\item \rbibentry{Hal09}          % Predictive Decoding of Neural Data
\end{bibenum}
\vspace{1em}

%% non-peerreviewed
%\begin{bibenum}
%\item \rbibentry{HH09}           % Advancing Neuroimaging Research with Predictive ...
%
%% posters
%\item \rbibentry{CWH+11}         % Scanning parameters for optimal decoding using a 32-channel head coil for fMRI
%\item \rbibentry{HHH10}          % Improving efficiency in cognitive neuroscience research with NeuroDebian
%\item \rbibentry{HMH+04}         % Structural Equation Modeling of Neuroimaging Data: ...
%\item \rbibentry{HHP04}          % Fusion of Functional Brain Imaging Modalities ...
%
%% conference proceedings
%\item \rbibentry{TMH+11}         % Debian Med: Integrated software environment 
%\end{bibenum}

\subsection{CHAPTERS}

\begin{bibenum}[ref=\arabic*,parsep=-0.4em]
\item \rbibentry{HHP05}          % Advanced Image Processing in Magnetic Resonance Imaging: ...
\end{bibenum}
\vspace{1em}

\subsection{INTERVIEWS}
\sloppy
\begin{bibenum}[ref=\arabic*,parsep=-0.4em]
\item \rbibentry{SFN13}
\item \rbibentry{FLOSS4SCIENCE11}
\item \rbibentry{INCF11}
\end{bibenum}
\vspace{1em}


%---Talks and Symposia
\section{Invited Talks}

% not detailed one
% \tldatecventry{2014}{PyCon 2014}{Montr\'eal, Canada}{}{}{}
% \tldatecventry{2014}{SRI International}{Menlo Park, CA}{}{}{}
% \tldatecventry{2013}{SEA Software Engineering Conference}{UCAR}{Boulder, CO}{}{}
% \tldatecventry{2013}{University of Pennsylvania}{Philadelphia, PA}{}{}{}
% \tldatecventry{2012}{INCF Bootcamp 2012}{Munich, Germany}{}{}{}
% \tldatecventry{2011}{Annual Meeting of Society for Neuroscience}{Washington, DC}{}{}{}
% \tldatecventry{2011}{EuroSciPy satellite ``Python in Neuroscience''}{Paris, France}{}{}{}
% \tldatecventry{2011}{EuroSciPy}{Paris, France}{}{}{}
% \tldatecventry{2011}{DebConf10}{New York, NY}{}{}{}
% \tldatecventry{2009}{Dartmouth College}{Hanover, NH}{}{}{}
% \tldatecventry{2009}{Berkeley}{CA}{}{}{}
% \tldatecventry{2009}{University of Hawaii at Manoa}{Honolulu, HI}{}{}{}

%\tldatecventry{2014}{PyCon 2014}{Fail2Ban -- keep your boxes skiddie-free}{}{Montr\'eal, Canada}{}
\tldatecventry{2014}{SRI International}{From statistical learning to an open-source, turnkey platform for neuroimaging}{}{Menlo Park, CA}{}
\tldatecventry{2013}{SEA Software Engineering Conference}{Open is not enough: benefits from Debian as an integrated, community-driven computing platform}{}{UCAR,Boulder, CO}{\phttp{sea.ucar.edu/event/open-not-enough-benefits-debian-integrated-community-driven-computing-platform}}
\cventry{}{University of Pennsylvania}{Environments for efficient contemporary research in neuroimaging}{}{Philadelphia, PA}{}
\tldatecventry{2012}{INCF Bootcamp 2012}{Applied NeuroDebian: Python in Neuroimaging}{}{Munich, Germany}{}
% \tldatecventry{2011}{Annual Meeting of Society for Neuroscience}{Multivariate analysis strategies of neuroimaging data in PyMVPA}{}{Washington, DC}{}
% \tldatecventry{2011}{EuroSciPy satellite ``Python in Neuroscience''}{The virtues and sins of PyMVPA}{}{Paris, France}{}
\tldatecventry{2011}{EuroSciPy}{$\pi$`s in Debian or Scientific Debian: NumPy, SciPy and beyond}{}{Paris, France}{}
% \tldatecventry{2011}{DebConf10}{Debian: The ultimate platform for neuroimaging research}{}{New York, NY}{}
% \tldatecventry{2009}{Dartmouth College}{An ecosystem of neuroimaging, statistical learning, and open-source software to make research more efficient, more open, and more fun}{}{Hanover, NH}{}
\tldatecventry{2009}{UC Berkeley}{Reliable Decoding of Neural Data}{}{Berkeley, CA}{}
\cventry{}{University of Hawaii at Manoa}{PyMVPA: Fathom Brain Function through Multivariate Pattern Analysis}{}{Honolulu, HI}{}

\section{Didactic Activities} % Teaching \& Lecturing}

\tlcventry{2012}{0}{}{PBS Department, Dartmouth College}{Consulting undergraduate and graduate students in
  application of statistical learning methodologies in their
  neuroimaging-based research}{}{}
\tllabelcventry{2010}{0}{}{Co-lecturer}{PyMVPA Workshops}{}{}{
2014. Hanse-Wissenschaftskolleg Institute for Advanced Study, Delmenhorst Germany\\
2012. Center for Behavioral Brain Sciences, Magdeburg Germany \\
2010. Psychology and Brain Sciences, Dartmouth College, Hanover USA
}
% TODO: TA for Jim's 2013 Fall MVPA class?
%\tldatecventry{2012}{Co-lecturer}{PyMVPA Workshop}{}{}{Center for Behavioral Brain Sciences, Magdeburg Germany}
%\tldatecventry{2010}{Co-lecturer}{PyMVPA Workshop}{}{}{Psychology and Brain Sciences, Dartmouth College, Hanover USA}
\tldatecventry{2000}{Teaching Assistant}{Intermediate Programming
  (CS251)}{}{}{\href{mailto:TODO}{Prof. David Ackley}, Computer Science Department, UNM}


% \section{Professional Community Activities}
% moved to Service and Outreach

%---Awards and fellowships
\section{Awards, Honors \& Fellowships}
\tldatecventry{1998}{Fellow}{The International Scientific Fund Representatives in Ukrainian Studentship Award}{}{}{}
\tldatecventry{1996}{Award}{}{}{The Academy of Sciences of Ukraine}{Project: \emph{Information-Measuring System With Optical Transformation Biomedical Information}}{}
\tldatecventry{1995}{Fellow}{The International Soros Science Educational Program (ISSEP) Studentship Award}{}{}{}
          \cventry{}{6th place}{}{}{ACM South-Eastern European Regional Programming Contest}{1st place at VSTU}
          \cventry{}{4th place}{Physics Contest among Colleges and Universities of Ukraine}{}{}{1st place at VSTU}
\tldatecventry{1994}{1st place}{Regional Programming Contest}{}{}{}
\tldatecventry{1993}{3rd place}{Regional Physics Contest}{}{}{}



% Outside interests/activities
\section{Extra Qualifications}

\cvitem{Languages}{Fluent in Russian, Ukrainian and English.}
\cvitem{Hobbies}{Major contributor to the
  \href{http://neuro.debian.net/coffeeart.html}{Coffee Art Collection}}

\end{document}

%%% Local Variables:
%%% mode: latex
%%% TeX-PDF-mode: t
%%% TeX-master: t
%%% ispell-local-dictionary: "american"
%%% auto-fill-inhibit-regexp: ".*[&|].*[&|].*"
%%% End:
